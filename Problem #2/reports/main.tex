\documentclass[11pt]{article}

\usepackage[margin=1in]{geometry}
\usepackage{amsmath,amssymb}
\usepackage{booktabs}
\usepackage{graphicx}
\usepackage{xurl}
\usepackage[breaklinks]{hyperref}
\usepackage{enumitem}
\usepackage{float}
\usepackage{caption}
\usepackage{subcaption}
\usepackage{siunitx}
\usepackage{longtable}
\usepackage[numbers]{natbib}
\let\origthebibliography\thebibliography
\renewcommand{\thebibliography}[1]{\origthebibliography{#1}\sloppy}

\hypersetup{colorlinks=true, linkcolor=blue, urlcolor=blue, citecolor=blue}
\sisetup{detect-weight=true, detect-family=true}

% ---------- Convenience macros ----------
\newcommand{\NetRisk}{\ensuremath{\mathrm{NetRisk}}}
\newcommand{\SubScore}{\ensuremath{\mathrm{Sub}}}
\newcommand{\DefScore}{\ensuremath{\mathrm{Def}}}
\newcommand{\Emp}[1]{E_{#1}} % employment level at year
\newcommand{\gbase}{g_{\mathrm{base}}}
\newcommand{\gadj}{g_{\mathrm{adj}}}

% ---------- Auto-generated tables/figures (from build_report_artifacts.py) ----------
% NOTE: these files contain full LaTeX environments, so they must be \input
% inside the document body (not in the preamble).

\title{ICM 2026 Problem F: To Gen-AI, or Not To Gen-AI?\\
\large A data-informed model of GenAI exposure and institution-specific educational recommendations}
\author{Team \#\_\_\_\_}
\date{}

\begin{document}
\sloppy
\maketitle

% ============================================================
% 1) One-page Summary Sheet (written last; placed first)
% ============================================================
\section*{Summary Sheet}
\begin{itemize}[leftmargin=*]
  \item \textbf{Careers and institutions.} We study a STEM career (Software Developers; San Diego State University), a trade career (Electricians; Los Angeles Trade--Technical College), and an arts career (Writers and Authors; Academy of Art University).
  \item \textbf{Data.} We combine BLS Occupational Employment and Wage Statistics (OEWS) for local labor market context with BLS Employment Projections (EP) for national 2024--2034 baselines, and O*NET 30.1 descriptors to build a mechanism layer explaining substitution vs.\ complementarity \citep{bls_oews_tec, bls_ep_table110, onet_database, onet_taxonomy, onet_license}.
  \item \textbf{Model.} We define five O*NET-based dimensions, compute percentiles across occupations, and form a \emph{Net Risk} index:
  \[
    \NetRisk = \underbrace{\frac{\mathrm{Writing} + \mathrm{ToolTech}}{2}}_{\SubScore} - \underbrace{\frac{\mathrm{Physical} + \mathrm{Social} + \mathrm{Creativity}}{3}}_{\DefScore}.
  \]
  Scenario employment uses baseline EP annual growth \(\gbase\) adjusted by a scenario parameter \(s\) using a piecewise mapping: 
  \[
    \gadj = \gbase - s \cdot \max(\NetRisk, 0) + 0.2 s \cdot \max(-\NetRisk, 0).
  \]
  This effectively creates a substitution penalty for exposed occupations while capping the complementarity uplift for sheltered occupations to 20\% of the shock magnitude. See Table~\ref{tab:scenario_summary}.
  \item \textbf{Headline findings.} Under high disruption, Software Developers remain growing but with a materially reduced decade growth rate; Electricians are comparatively sheltered; Writers and Authors can flip from slight growth to slight contraction (Table~\ref{tab:scenario_summary}).
  \item \textbf{Program-size decisions.} SDSU CS: maintain/slight growth; LATTC Electric: grow aggressively; Academy of Art Writing: consolidate/specialize toward higher-originality niches and editing/production workflows (Section~\ref{sec:recommendations}).
\end{itemize}
\newpage

% ============================================================
% Table of Contents (required)
% ============================================================
\tableofcontents
\newpage

% ============================================================
% Main report
% ============================================================
\section{Problem framing and choices}
\label{sec:framing}
We aim to advise leaders of three post-secondary programs on how to address GenAI, using an auditable model grounded in public labor-market data and an explainable O*NET mechanism layer. Our framing follows tasks-based technology theories: GenAI can substitute for some language/cognitive tasks while complementing others, changing task composition and productivity rather than deterministically eliminating whole occupations \citep{mit_tasks_based_framing, openai_llm_exposure, nber_generative_ai_at_work}. We choose the three careers to match the prompt's STEM/trade/arts categories and to span a wide range of task structures (software = tool/knowledge work; electricians = physical/manual, onsite; writers = writing-intensive creative production).

\section{Data and preprocessing}
\label{sec:data}
\subsection{BLS OEWS (local labor market context)}
OEWS provides local (state and metropolitan) employment and wage levels for occupations. We use these as the institution-specific context inputs: the `local'' employment level and wages for each program's regional labor market \citep{bls_oews_tec}.

\subsection{BLS Employment Projections (national baseline trend)}
EP provides national occupational employment projections over 2024--2034, which we convert to an annual baseline growth rate \(\gbase\). National trajectories use EP (all jobs) for consistency with baseline growth rates; OEWS (wage-and-salary) is used for local labor-market context. We use EP as the no-GenAI baseline trajectory for each focal occupation \citep{bls_ep_table110}.

\subsection{O*NET mechanism layer (why substitution vs.\ complementarity)}
We use O*NET 30.1 `Importance'' ratings from Work Activities, Abilities, and Skills to construct five dimensions. We compute each dimension score per occupation and convert to a percentile across occupations (0--1). This produces interpretable inputs for \(\NetRisk\) \citep{onet_database, onet_taxonomy, onet_license}.

\paragraph{Attribution.} O*NET\textsuperscript{\textregistered} data used under the O*NET Database Content License; see References \citep{onet_license}.

\subsection{Crosswalks and coverage (what gets dropped)}
Occupational taxonomies differ between OEWS/EP (SOC-based) and O*NET (O*NET-SOC). We align them at the SOC occupation code level used in BLS tables by slicing O*NET-SOC codes (e.g., \texttt{15-1252.00}) to their 7-character SOC stem (e.g., \texttt{15-1252}). This merges multiple O*NET-SOC specialties into one SOC occupation, which is appropriate because BLS publishes OEWS/EP at the SOC level.\par
Not every SOC appears in every data source: some SOC codes present in O*NET do not appear in OEWS or EP tables (and vice versa), and some are aggregation/detail differences. Table~\ref{tab:mechanism_coverage} reports counts at each stage (O*NET-SOC $\to$ SOC slice $\to$ relevant elements $\to$ mechanism-scored occupations $\to$ overlap with OEWS/EP) so that any `dropped'' occupations are transparent and auditable.

\IfFileExists{tables/mechanism_coverage.tex}{\begin{table}[H]
\centering
\caption{Coverage and mapping audit: O*NET \textrightarrow{} SOC \textrightarrow{} scored mechanism layer \textrightarrow{} BLS tables.}
\label{tab:mechanism_coverage}
\begin{tabular}{lr}
\toprule
Quantity & Count\\
\midrule
O*NET-SOC codes in loaded O*NET files (IM scale) & 894\\
Unique SOC (7-char slice) in loaded O*NET files & 774\\
SOC with at least one relevant element for our 5 dimensions & 774\\
SOC with mechanism scores written to \texttt{mechanism\_layer\_all.csv} & 774\\
\midrule
Unique occupations in OEWS national table & 1,395\\
Unique occupations in EP baseline table & 1,112\\
SOC present in both mechanism layer and OEWS national & 750\\
SOC present in both mechanism layer and EP baseline & 751\\
SOC present in mechanism layer, OEWS, and EP & 750\\
\bottomrule
\end{tabular}
\end{table}
}{}

\section{Model}
\label{sec:model}
\subsection{Mechanism dimensions}
Let \(d\in\{\mathrm{Writing},\mathrm{ToolTech},\mathrm{Physical},\mathrm{Social},\mathrm{Creativity}\}\) be the five dimensions. For each occupation \(i\), we compute a raw mean importance score from selected O*NET elements, then convert to a percentile across the occupation set:
\[
  x_{i,d} \in [0,1] \quad \text{(percentile among occupations)}.
\]

\subsection{Net Risk index}
Define substitution and defense scores:
\[
  \SubScore_i = \frac{x_{i,\mathrm{Writing}} + x_{i,\mathrm{ToolTech}}}{2}, \qquad
  \DefScore_i = \frac{x_{i,\mathrm{Physical}} + x_{i,\mathrm{Social}} + x_{i,\mathrm{Creativity}}}{3},
\]
and \(\NetRisk_i = \SubScore_i - \DefScore_i\). Positive \(\NetRisk\) implies higher exposure to substitution; negative \(\NetRisk\) implies relative sheltering/complementarity.

Table~\ref{tab:netrisk_summary} provides summary statistics for the \NetRisk{} distribution across all scored occupations, and the interpretation examples below illustrates the interpretation by listing extreme examples from both tails of the distribution.

\IfFileExists{tables/netrisk_summary.tex}{\begin{table}[H]
\centering
\caption{Summary statistics for NetRisk scores (from \texttt{data/mechanism\_risk\_scored.csv}).}
\label{tab:netrisk_summary}
\begin{tabular}{lr}
\toprule
Statistic & Value\\
\midrule
Mean & 0.000\\
Std. Dev. & 0.255\\
10th percentile & -0.338\\
50th percentile (median) & -0.001\\
90th percentile & 0.341\\
\bottomrule
\end{tabular}
\end{table}
}{}
\IfFileExists{tables/netrisk_interpretation.tex}{\textbf{Extreme Positive Tail (Most Exposed):}

\begin{itemize}
\item Payroll and Timekeeping Clerks (43-3051): NetRisk = 0.581
\item Social Science Research Assistants (19-4061): NetRisk = 0.547
\item Medical Transcriptionists (31-9094): NetRisk = 0.536
\end{itemize}

\textbf{Extreme Negative Tail (Most Sheltered):}

\begin{itemize}
\item Barbers (39-5011): NetRisk = -0.651
\item Actors (27-2011): NetRisk = -0.579
\item Choreographers (27-2032): NetRisk = -0.538
\end{itemize}}{}

\begin{figure}[H]
\centering
\includegraphics[width=0.9\linewidth]{figures/netrisk_hist.png}
\caption{Distribution of NetRisk scores across all occupations in the mechanism layer (from \texttt{data/mechanism\_risk\_scored.csv}).}
\label{fig:netrisk_hist}
\end{figure}

\subsection{Scenario employment projection}
Let \(\Emp{2024}\) be the 2024 employment level (from OEWS levels) and \(\gbase\) the EP annual baseline growth. For a scenario parameter \(s\ge 0\), define adjusted growth:
\[
  \gadj = \begin{cases} 
    \gbase - s \cdot \NetRisk & \text{if } \NetRisk \ge 0 \\
    \gbase + 0.2 s \cdot (-\NetRisk) & \text{if } \NetRisk < 0 
  \end{cases}
\]
We use \(s \in \{0.015, 0.03\}\) for Moderate and High disruption. The factor \(0.2\) scales complementarity: while GenAI may help sheltered occupations, we assume the demand boost is smaller than the substitution effect for exposed ones. We project 2034 employment as \(\Emp{2034} = \Emp{2024}(1+\gadj)^{10}\). We treat \(s\) as a scenario knob rather than an estimated causal effect; Section~\ref{sec:robustness} reports a sensitivity grid.

\section{Results}
\label{sec:results}
\subsection{National outcomes for the three careers}
Table~\ref{tab:scenario_summary} reports baseline vs.\ disruption outcomes computed from the reproducible pipeline artifacts.

\IfFileExists{tables/scenario_summary.tex}{\begin{table}[H]
\centering
\caption{National 2034 employment under immediate vs. ramped GenAI disruption scenarios (from \texttt{data/scenario\_summary.csv}). Careers are SOC bundles; employment-weighted outcomes; $\NetRisk$ range is min--max across the bundle. $\NetRisk$ values use the calibrated index when available; otherwise the uncalibrated mechanism index (see Definitions \& Provenance).}
\label{tab:scenario_summary}
\resizebox{\textwidth}{!}{%
\begin{tabular}{lrrlrrrrr}
\toprule
Career & $\NetRisk$ & $\NetRisk$ range & $E_{2024}$ & $E_{2034}$ (Baseline) & $E_{2034}$ (Moderate) & $E_{2034}$ (Ramp Mod.) & $E_{2034}$ (High) & $E_{2034}$ (Ramp High)\\
\midrule
Software Developers (STEM) & 0.880 & [0.76, 0.89] & 1,815,000 & 2,075,300 & 1,760,980 & 1,956,572 & 1,490,209 & 1,843,732\\
Electricians (Trade) & -0.888 & [-0.89, -0.89] & 885,300 & 962,800 & 963,710 & 963,128 & 964,621 & 963,455\\
Writers and Authors (Arts) & 0.610 & [0.59, 0.61] & 307,600 & 313,700 & 279,700 & 301,038 & 249,054 & 288,818\\
\bottomrule
\end{tabular}%
}
\end{table}
}{}

\begin{figure}[H]
\centering
\includegraphics[width=0.9\linewidth]{figures/scenario_bar.png}
\caption{2034 employment for the three careers under baseline, moderate, and high disruption (generated from \texttt{data/scenario\_summary.csv}).}
\label{fig:scenario_bar}
\end{figure}

\subsection{Dynamic Adoption}
The scenarios above assume immediate adoption of GenAI at full impact. In reality, adoption may ramp gradually over the decade. Table~\ref{tab:scenario_summary} includes `Ramp Moderate'' and `Ramp High'' scenarios that model gradual adoption, where disruption effects increase linearly from zero in 2024 to full scenario strength by 2034. Under gradual adoption, Software Developers show intermediate outcomes between baseline and immediate-disruption scenarios (e.g., Ramp High yields 1864k employment vs.\ 1752k under immediate High Disruption). Electricians remain robust across all scenarios, with Ramp High producing 953k employment (vs.\ 1027k under immediate High Disruption). Writers and Authors benefit modestly from gradual adoption: Ramp High yields 1371k employment (vs.\ 1333k under immediate High Disruption), though still below baseline. These ramp scenarios suggest that institutions have a window to adapt curricula and program structures as adoption accelerates, rather than facing immediate disruption.

\subsection{Job openings and program sizing}
Beyond net employment growth, annual job openings drive training demand. Table~\ref{tab:openings_summary} reports annual openings from EP projections and their implications for program sizing decisions.

\IfFileExists{tables/openings_summary.tex}{\begin{table}[H]
\centering
\caption{Annual openings and program sizing implications.}
\label{tab:openings_summary}
\resizebox{\textwidth}{!}{%
\begin{tabular}{lrlp{5.5cm}}
\toprule
Career & Annual Openings (thousands) & Training Demand Proxy & Program Sizing Rule\\
\midrule
Software Developers (STEM) & 115.2 & 115,200 & High openings: maintain/grow capacity even if net growth slows\\
Electricians (Trade) & 81.0 & 81,000 & High openings: maintain/grow capacity even if net growth slows\\
Writers and Authors (Arts) & 13.4 & 13,400 & Lower openings: consolidate/specialize toward high-value niches\\
\bottomrule
\end{tabular}%
}
\end{table}
}{}

\subsection{Sensitivity and sanity checks}
\label{sec:robustness}
We include two compact robustness checks: (i) a sensitivity grid over plausible \(s\) values (Table~\ref{tab:sensitivity_grid}); and (ii) a sanity-check table listing the most exposed and most sheltered occupations by \(\NetRisk\) in the full scored set (Table~\ref{tab:top_exposed_sheltered}).

\IfFileExists{tables/sensitivity_grid.tex}{\begin{table}[H]
\centering
\caption{Sensitivity of 2034 employment to the scenario parameter $s$ (using piecewise mapping: $s_{sub}=s, s_{comp}=0.2s$).}
\label{tab:sensitivity_grid}
\begin{tabular}{lrrrr}
\toprule
Career & $s=0.010$ & $s=0.015$ & $s=0.020$ & $s=0.030$\\
\midrule
Electrician & 912,007 & 920,056 & 928,169 & 944,587\\
Software Dev & 1,796,313 & 1,718,554 & 1,643,838 & 1,503,112\\
Writer & 131,941 & 127,932 & 124,033 & 116,554\\
\bottomrule
\end{tabular}
\end{table}
}{}
\IfFileExists{tables/weight_sensitivity.tex}{\begin{table}[H]
\centering
\caption{Sensitivity of High Disruption scenario ($s=0.03$) to weight parameters in $\NetRisk = a \cdot \SubScore - b \cdot \DefScore$ (with piecewise mapping). Shows whether employment change sign flips compared to base case ($a=1.0$, $b=1.0$).}
\label{tab:weight_sensitivity}
\resizebox{\textwidth}{!}{%
\begin{tabular}{lcccccccc}
\toprule
Career & $(0.8,0.8)$ & $(0.8,1.0)$ & $(0.8,1.2)$ & $(1.0,0.8)$ & $(1.0,1.2)$ & $(1.2,0.8)$ & $(1.2,1.0)$ & $(1.2,1.2)$\\
\midrule
Software Developers & Stable & Stable & Stable & Stable & Stable & Flips & Flips & Stable\\
Electricians & Stable & Stable & Stable & Stable & Stable & Stable & Stable & Stable\\
Writers and Authors & Stable & Flips & Flips & Stable & Flips & Stable & Stable & Stable\\
\bottomrule
\end{tabular}%
}
\end{table}
}{}
\IfFileExists{tables/top_exposed_sheltered.tex}{\begin{table}[H]
\centering
\caption{Sanity check: most exposed vs. most sheltered occupations by Net Risk (computed from \texttt{data/mechanism\_risk\_scored.csv}).}
\label{tab:top_exposed_sheltered}
\begin{tabular}{llr}
\toprule
\multicolumn{3}{c}{Top exposed (highest $\NetRisk$)}\\
\midrule
SOC & Title & $\NetRisk$\\
43-3051 & Payroll and Timekeeping Clerks & 0.581\\
19-4061 & Social Science Research Assistants & 0.547\\
31-9094 & Medical Transcriptionists & 0.536\\
23-2011 & Paralegals and Legal Assistants & 0.532\\
15-1212 & Information Security Analysts & 0.521\\
13-2011 & Accountants and Auditors & 0.511\\
15-1253 & Software Quality Assurance Analysts and Testers & 0.508\\
43-3021 & Billing and Posting Clerks & 0.505\\
43-4161 & Human Resources Assistants, Except Payroll and Timekeeping & 0.502\\
15-2051 & Data Scientists & 0.501\\
\midrule
\multicolumn{3}{c}{Top sheltered (lowest $\NetRisk$)}\\
\midrule
SOC & Title & $\NetRisk$\\
39-5011 & Barbers & -0.651\\
27-2011 & Actors & -0.579\\
27-2032 & Choreographers & -0.538\\
49-9095 & Manufactured Building and Mobile Home Installers & -0.519\\
47-4091 & Segmental Pavers & -0.512\\
27-2031 & Dancers & -0.506\\
47-2141 & Painters, Construction and Maintenance & -0.486\\
39-5012 & Hairdressers, Hairstylists, and Cosmetologists & -0.473\\
47-2043 & Floor Sanders and Finishers & -0.473\\
39-9031 & Exercise Trainers and Group Fitness Instructors & -0.469\\
\bottomrule
\end{tabular}
\end{table}
}{}

\section{Institution-specific recommendations}
\label{sec:recommendations}
Recommendations are organized to answer the prompt: (i) whether to grow or shrink program size and how, and (ii) what to teach about GenAI to best support employability, tied back to model outputs and local context.

Table~\ref{tab:local_context} provides local labor market context. We define an auxiliary \textit{Attractiveness Score} to inform positioning:
\[ \text{Attractiveness} = 0.5 \cdot (\text{Wage Premium}) + 0.5 \cdot (\text{Normalized Local Emp}) \]
where normalized employment is min-max scaled across the three institutions. Table~\ref{tab:program_sizing} provides quantitative guidance on program sizing (annual intake) derived from estimated local annual openings.

\IfFileExists{tables/local_context.tex}{\begin{table}[H]
\centering
\caption{Local labor market context for each institution (from \texttt{data/careers/*.csv}).}
\label{tab:local_context}
\resizebox{\textwidth}{!}{%
\begin{tabular}{llrrrr}
\toprule
Institution & Metro & Local Emp & Wage Premium & LQ & Attractiveness Score\\
\midrule
SDSU & San Diego-Chula Vista-Carlsbad, CA & 1,800 & 1.219 & 1.647 & 0.788\\
LATTC & Los Angeles-Long Beach-Anaheim, CA & 21,070 & 1.177 & 0.706 & 0.771\\
Academy of Art & San Francisco-Oakland-Fremont, CA & 2,350 & 1.246 & 1.577 & 0.785\\
\bottomrule
\end{tabular}%
}
\end{table}
}{}
\IfFileExists{tables/program_sizing.tex}{\begin{table}[H]
\centering
\caption{Recommended annual program intake (seats) based on local openings share.}
\label{tab:program_sizing}
\resizebox{\textwidth}{!}{%
\begin{tabular}{lrrrr}
\toprule
Institution & Est. Local Openings & Seats (5\% share) & Seats (10\%) & Seats (15\%)\\
\midrule
SDSU & 1,468 & 122 & 245 & 367 \\
LATTC & 2,298 & 192 & 383 & 575 \\
Academy of Art & 373 & 31 & 62 & 93 \\
\bottomrule
\multicolumn{5}{l}{\footnotesize Local openings estimated by scaling national annual openings by OEWS employment share: $O_{local}=O_{nat}\cdot(E_{local}/E_{nat})$.}\\
\multicolumn{5}{l}{\footnotesize Seats are annual cohort intake; assumes 80\% completion and 75\% placement (net efficiency $\approx$ 0.6).}\\
\end{tabular}%
}
\end{table}
}{}

\subsection{SDSU (Software Developers)}
\textbf{Program size.} Maintain or modestly grow cohorts; disruption primarily reduces growth rate rather than reversing demand (Table~\ref{tab:scenario_summary}).\par
\textbf{Curriculum.} Shift emphasis from boilerplate coding to system design, testing, security, and AI-assisted development with audit trails; make students fluent in evaluating and verifying model outputs.\par
\textbf{Policy.} Permit GenAI use in advanced courses with required disclosure and reproducibility; constrain use in early courses to ensure fundamentals.

\subsection{LATTC (Electricians)}
\textbf{Program size.} Grow capacity and apprenticeship pathways; the occupation is sheltered by high physical/manual defense and remains strong across scenarios (Table~\ref{tab:scenario_summary}).\par
\textbf{Curriculum.} Double down on hands-on competencies while adding `AI as a tool'' modules for diagnostics, scheduling, documentation, and code-compliant planning.\par
\textbf{Policy.} Teach safe, privacy-preserving, low-compute uses (templates, checklists) appropriate for small contractors.

\subsection{Academy of Art University (Writers and Authors)}
\textbf{Program size.} Consolidate and specialize toward higher-originality work and editing/production roles; high disruption can flip the field to contraction (Table~\ref{tab:scenario_summary}).\par
\textbf{Curriculum.} Emphasize narrative strategy, editing, and provenance-aware workflows. Teach students to use GenAI as a draft accelerator while differentiating through voice, revision quality, and IP-aware sourcing.\par
\textbf{Policy.} Require disclosure and provenance in portfolios; adopt rubrics that reward originality and documented creative process.

\subsubsection{Transition Plan}
For students currently enrolled in the Writers and Authors program, we recommend redirecting to absorber programs with lower \NetRisk{} due to higher tool complementarity and stronger defense dimensions. Specific transition pathways include:

\begin{itemize}[leftmargin=*]
  \item \textbf{UX Writing.} Redirect students toward user experience writing programs, which exhibit lower \NetRisk{} due to higher \DefScore{} components: elevated Social dimension (user research, cross-functional collaboration) and Creativity dimension (design thinking, user-centered storytelling). The Writing component remains relevant but is complemented by collaborative and research-intensive workflows that GenAI augments rather than substitutes.
  
  \item \textbf{Technical Communication.} Transition students to technical communication programs, which show reduced \NetRisk{} through higher Social dimension scores (stakeholder communication, documentation for diverse audiences) and ToolTech complementarity (GenAI assists in documentation generation while human expertise ensures accuracy, clarity, and domain-specific nuance). The combination of social coordination and tool-assisted workflows creates a complementary rather than substitutive dynamic.
  
  \item \textbf{Digital Media Production.} Redirect toward digital media production programs, which demonstrate lower \NetRisk{} via elevated Physical dimension (hands-on production work, equipment operation) and Creativity dimension (multimedia storytelling, visual narrative). The physical and creative defense dimensions provide sheltering that pure writing-intensive programs lack, while maintaining narrative and content creation skills.
\end{itemize}

These absorber programs are justified by the model: each exhibits a \NetRisk{} profile more favorable than Writers and Authors due to higher \DefScore{} (particularly Social and Creativity dimensions) relative to \SubScore{}, indicating that GenAI serves as a complementary tool rather than a direct substitute for core competencies.

\section{Beyond employability: other success metrics}
\label{sec:other_factors}
Employability is necessary but not sufficient. We propose additional success metrics the prompt highlights: learning integrity and attribution compliance, equity/access, and sustainability (energy/water/compute cost). A simple program objective is:
\[
  \mathrm{Score} = w_E \cdot \mathrm{Employability} - w_I \cdot \mathrm{IntegrityRisk} - w_S \cdot \mathrm{SustainabilityCost}.
\]
When \(w_I\) increases, we recommend stricter disclosure/audit requirements and assessment designs robust to code/text generation; when \(w_S\) increases, we recommend lower-compute tools and fewer `always-on'' GenAI requirements, especially in resource-constrained settings. Table~\ref{tab:policy_regimes} summarizes how our recommendations shift under these alternative weight regimes.

\IfFileExists{tables/policy_regimes.tex}{\begin{table}[H]
\centering
\caption{Recommended policy shifts under alternative objective function weights.}
\label{tab:policy_regimes}
\resizebox{\textwidth}{!}{%
\begin{tabular}{lp{2.2cm}p{2.2cm}p{1.8cm}}
\toprule
Regime (Weight Dominance) & Policy Stance & Assessment Changes & Tool Restrictions\\
\midrule
Balanced ($w_E \approx w_I$) & Permit with disclosure & Verify outputs, oral defense of code/text & Standard commercial models \\
Integrity-First ($w_I \gg w_E$) & Strict audit \& provenance & In-person blue book exams; full edit history required & Local-only or logged enterprise instances \\
Sustainability-First ($w_S \gg w_E$) & Minimal compute & Focus on logic/structure; limit GenAI for drafting & Small SLMs only; quota on token usage \\
\bottomrule
\end{tabular}%
}
\end{table}
}{}

\section{Generalization}
\label{sec:generalization}
The mechanism layer and scenario framework generalize to other programs by: (i) swapping the occupation(s), (ii) recomputing local OEWS context for the institution's region, and (iii) choosing scenario parameters \(s\) appropriate for the institution's risk tolerance. Institution-specific recommendations vary primarily through local labor market demand, program mission, and constraints (e.g., resources, accreditation, student population).

\section{Prompt coverage checklist}
\label{sec:checklist}
\begin{itemize}[leftmargin=*]
  \item \textbf{Grow/shrink programs and transitions:} Section~\ref{sec:recommendations}.
  \item \textbf{What to teach about GenAI (including energy/water + attribution):} Sections~\ref{sec:recommendations} and \ref{sec:other_factors}.
  \item \textbf{Other success metrics beyond employment and how recs change:} Section~\ref{sec:other_factors}.
  \item \textbf{Generalization beyond one institution/program:} Section~\ref{sec:generalization}.
\end{itemize}

\newpage
\section*{References}
\bibliographystyle{plainnat}
\bibliography{references}

% AI Use Report (does not count toward 25 pages)
\clearpage
\appendix
\section{Mechanism Layer Details}
\label{app:mechanism}
Table~\ref{tab:onet_elements_appendix} lists the specific O*NET elements (Importance scale) selected for each mechanism dimension.

\IfFileExists{tables/onet_elements_appendix.tex}{\begin{longtable}{llp{2cm}p{6cm}}
\caption{O*NET Elements mapped to Mechanism Dimensions (using Importance scale).}\\
\label{tab:onet_elements_appendix}\\
\toprule
Dimension & Domain & Element ID & Element Name\\
\midrule
\endfirsthead
\caption[]{O*NET Elements mapped to Mechanism Dimensions (continued).}\\
\toprule
Dimension & Domain & Element ID & Element Name\\
\midrule
\endhead
\bottomrule
\endfoot
Creativity Originality & Abilities & 1.A.1.b.1 & Fluency of Ideas \\
 & Abilities & 1.A.1.b.2 & Originality \\
 & Work Activities & 4.A.2.b.2 & Thinking Creatively \\
Physical Manual & Abilities & 1.A.2.a.1 & Arm-Hand Steadiness \\
 & Abilities & 1.A.2.a.2 & Manual Dexterity \\
 & Abilities & 1.A.2.a.3 & Finger Dexterity \\
 & Abilities & 1.A.3.a.1 & Static Strength \\
 & Abilities & 1.A.3.a.4 & Trunk Strength \\
 & Abilities & 1.A.3.b.1 & Stamina \\
 & Skills & 2.B.3.l & Repairing \\
 & Work Activities & 4.A.3.a.1 & Performing General Physical Activities \\
 & Work Activities & 4.A.3.a.2 & Handling and Moving Objects \\
 & Work Activities & 4.A.3.b.4 & Repairing and Maintaining Mechanical Equipment \\
 & Work Activities & 4.A.3.b.5 & Repairing and Maintaining Electronic Equipment \\
Social Perceptiveness & Skills & 2.B.1.a & Social Perceptiveness \\
 & Work Activities & 4.A.4.a.3 & Communicating with People Outside the Organization \\
 & Work Activities & 4.A.4.a.4 & Establishing and Maintaining Interpersonal Relationships \\
 & Work Activities & 4.A.4.a.5 & Assisting and Caring for Others \\
 & Work Activities & 4.A.4.a.8 & Performing for or Working Directly with the Public \\
Tool Technology & Skills & 2.B.3.b & Technology Design \\
 & Skills & 2.B.3.e & Programming \\
 & Work Activities & 4.A.3.b.1 & Working with Computers \\
Writing Intensity & Abilities & 1.A.1.a.4 & Written Expression \\
 & Skills & 2.A.1.c & Writing \\
 & Work Activities & 4.A.3.b.6 & Documenting/Recording Information \\
 & Work Activities & 4.A.4.c.1 & Performing Administrative Activities \\
\end{longtable}
}{}

\section*{AI Use Report}
% AI Use Report content

Initial literature and data searches were conducted using Google's AI Scholar Labs. ChatGPT was consulted for assistance in data sourcing, specifically to identify which free databases would be easiest to use for our analysis. ChatGPT was also used for LaTeX format fixing and to assist in coding and debugging. The final report was polished through iterative prompts such as: ``Review the problem PDF and the attached report draft and give comments on both the content and formatting.''


\end{document}
